% !TeX root = ..\protokoll.tex
\documentclass[../protokoll.tex]{subfiles}
\graphicspath{{\subfix{../images/}}}
\begin{document}
\part{Einleitung}
In diesem Versuch wird die Fraunhofer und Fresnel-Beugung, Interferenz behandelt. In diesem wird z.B. ein lichtundurchlässiger Schirm genommen und beleuchtet, so wird durch die geometrische Position des Schirms ein Schatten erwartet, dennoch lässt sich gut erkennen, dass sich hingegen der geometrischen Position doch Licht in den erwarteten Schattenbereich eingedrungen ist. Erklären lässt sich dies durch die Beugung von Lichtwellen an Hindernissen und die entstandenen Beugungsbilder lassen sich durch das Huygensschen Prinzips und Interferenz von Licht erklären. Des Weiteren wird sich feststellen lassen, dass bei der Arbeit mit optischen Instrumenten das Phänomen der Beugung sich als unerwünscht herausstellt, da sich nie ein Punkt abbilden lässt, sondern nur eine Beugung des Punktes.  
\end{document}