% !TeX root = ..\protokoll.tex
\documentclass[../protokoll.tex]{subfiles}
\graphicspath{{\subfix{../images/}}}
\begin{document}
\section{Bestimmung der Breite eines Beugunsspaltes}
In diesem Versuch wird die Breite eines Beugungsspaltes bestimmt. Dazu wird vor einem Laser ein Strahlaufweitungsswystem montiert, welches den Strahldurchmesser des Licht des Lasers (Wellenlänge $\lambda=632,8$ $nm$) vergrößert.Davor ist der zu untersuchende Beugungsspalt, welcher zuerst, mittels der davorliegenden Kamera, genau ausgerichtet wird, ehe dazwischen eine Linse mit Brennweite $f=(120\pm 2)$ $mm$ montiert wird. Die Kamera wird in der Brennebene der Linse platziert und zusätzlich wird ein Neutralfilterrad eingesetzt, um die Lichtintensität abzuschwächen, da die Messung mit der Kamera sonst ungenau wird.

Nun werden auf der Kamera die Intensitätsmaxima aufgenommen. wozu aufgrund der Auflösung mehrere Bilder nötig sind. Ein Pixel auf dem Bild entspricht einer Breite von $p=5,6$ $\mu m$.
Aus den Pixelpositionen x lassen sich mit $u=x\cdot p$ die räumlichen Positionen u der Maxima und Minima berechnen.
Das x des 0-te Maximums lässt sich aus den Positionen der $\pm$1-ten Minima bestimmen. Zusätzlich lässt sich aus der Beugungswinkel $\theta$ annähern als $\theta \approx \frac{u}{f}$. Zusammen mit den Formeln (18) und (20) aus dem Skript \cite{script} lassen sich für jede Extremstelle Werte für die Spaltbreite D berechnen. Für das 0-te Maximum ergeben sich somit folgende Werte: x=318 und u=1,78 mm.  In der Tabelle \ref{tab11} sind die Positionen, räumlichen Koordinaten und berechneten Spaltbreiten der ersten Maxima und Minima gegeben.

\begin{table}[h]
\centering
\begin{tabular}{|l|lll|lll|}
\hline
  & \multicolumn{3}{l|}{Minima}                                     & \multicolumn{3}{l|}{Maxima}                                     \\ \hline
n & \multicolumn{1}{l|}{x}   & \multicolumn{1}{l|}{u/mm}   & D/mm   & \multicolumn{1}{l|}{x}   & \multicolumn{1}{l|}{u/mm}   & D/mm   \\ \hline
1 & \multicolumn{1}{l|}{401} & \multicolumn{1}{l|}{2,2456} & 0,1631 & \multicolumn{1}{l|}{444} & \multicolumn{1}{l|}{2,4864} & 0,1538 \\ \hline
2 & \multicolumn{1}{l|}{487} & \multicolumn{1}{l|}{2,7272} & 0,1603 & \multicolumn{1}{l|}{524} & \multicolumn{1}{l|}{2,9344} & 0,1618 \\ \hline
3 & \multicolumn{1}{l|}{575} & \multicolumn{1}{l|}{3,22}   & 0,1582 & \multicolumn{1}{l|}{604} & \multicolumn{1}{l|}{3,3824} & 0,1645 \\ \hline
4 & \multicolumn{1}{l|}{654} & \multicolumn{1}{l|}{3,6624} & 0,1614 & \multicolumn{1}{l|}{688} & \multicolumn{1}{l|}{3,8528} & 0,164  \\ \hline
5 & \multicolumn{1}{l|}{727} & \multicolumn{1}{l|}{4,0712} & 0,1657 & \multicolumn{1}{l|}{757} & \multicolumn{1}{l|}{4,2392} & 0,1693 \\ \hline
6 & \multicolumn{1}{l|}{792} & \multicolumn{1}{l|}{4,4352} & 0,1716 & \multicolumn{1}{l|}{816} & \multicolumn{1}{l|}{4,5696} & 0,1765 \\ \hline
\end{tabular}
\caption{Darstellung der gemessenen Werte für x und der berechneten Werte für D}
\label{tab11}
\end{table}
Somit ergibt sich ein Mittelwert $\Bar{D}=0,1642$ $mm$ mit einer Standardabweichung von $\sigma=0,0058$ $mm$.
\end{document}