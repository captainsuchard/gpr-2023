% !TeX root = ..\protokoll.tex
\documentclass[../protokoll.tex]{subfiles}
\graphicspath{{\subfix{../images/}}}
\begin{document}
\part{Theorie}
Die elementaren Wissenselemente für die kommenden Versuche sollten aus der
Schule noch bekannt sein, jedoch werden diese hier nochmal kurz zusammengefasst.
Hierbei wird es um die Entstehung der verschiedenen Typen von Radioaktivität
gehen und um die beiden angewendeten Messmethoden 
(\textsc{Geiger-Müller-Zähler} und Szintillationsdetektor).

\section{Bezeichnungsweisen}
Um den nachfolgenden Abschnitt über die Schreibweise der Nukliden verstehen zu 
können, werden die folgenden Schreibweisen/Bezeichnungen engeführt:

\begin{table}[H]
    \caption{Benutzte Schreibweisen/Bezeichnungen in diesem Protokoll, entnommen aus \cite{script}, S. 30}
    \centering
    \begin{tabular}{|l|l|l|}
        \hline
        \textbf{Symbol} & \textbf{Einheit} & \textbf{Physikalische Größe} \\ \hline \hline
        $Z$ & & Ordnung- oder Kernladungszahl: Zahl der Protonen in einem Atomkern \\ \hline
        $N$ & & Zahl der Neutronen in einem Atomkern \\ \hline
        $A = Z + N$ & & Massenzahl \\ \hline
        $p$ & & Proton \\ \hline
        $n$ & & Neutron \\ \hline
        $\beta^-$ & & Elektron \\ \hline
        $\beta^+$ & & Positron \\ \hline
        $\alpha$ & & Alpha-Teilchen \\ \hline
        $v$ & & Neutrino \\ \hline
        $\mathbf{\bar{v}}$ & & Antineutrino \\ \hline
        $\gamma$ & & Gammaquant \\ \hline
        $h$ & \unit{\joule\second} & Plancksche Konstante: $h = \qty{6.62606957(29)e-34}{\joule\second}$ \\ \hline
        $f$ & \unit{\per\second} & Frequenz eines Gammaquants \\ \hline
        $E=hf$ & \unit{\joule}, \unit{\electronvolt} & Energie eines Gammaquants; $ \qty{1}{\electronvolt} \approx \qty{1.602e-19}{\joule}$ \\ \hline
        $T_{1/2}$ & \unit{\second} & Halbwertszeit \\ \hline
        $\lambda = \ln \frac{2}{T_{1/2}}$ & \unit{\per\second} & Zerfallskonstante \\ \hline
    \end{tabular}
\end{table}
\end{document}