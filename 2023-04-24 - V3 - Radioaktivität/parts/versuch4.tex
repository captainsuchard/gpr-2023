% !TeX root = ..\protokoll.tex
\documentclass[../protokoll.tex]{subfiles}
\graphicspath{{\subfix{../images/}}}
\begin{document}
\section{Abschwächung von \texorpdfstring{$\beta$}{Beta}-Strahlung in Materie}\label{sec:Abschwächung Beta-Strahlung}
Dieser Versuch verläuft ähnlich wie den der $\gamma$-Strahlung nur, dass hier statt variierende Eisenplatten Aluminiumplatten genommen werden, die wesentlich dünner sind als die vorher verwendeten Eisenplatten. Wobei die Dicke x der Platten auch hier wieder variiert und dass hier die Sr-90 Quelle verwendet wird und nicht die Cs-137-Quelle. Die Berechnung von n erfolgt analog zum vorherigen Versuch.   
\begin{figure}[H]
    \centering
    \includegraphics[width=0.7\textwidth]{2023-04-24 - V3 - Radioaktivität/images/Betta strahlung mit dem .png}
    \caption{$\beta$- Strahlung verlauf von n(x) über x halblogarithmisch aufgetragen}
    \label{Abb.2}
\end{figure}
Durch die Erstellung des nicht linearen Fits wird auch die Steigung bestimmt, welche hierbei eine Steigung von $0,90413 \pm 0,15651$ ergibt und somit den Abschwächung Koeffizienten von Aluminium angibt.
\begin{table}[H]
\centering
\begin{tabular}{|l|l|l|l|}
\hline
Dicke /mm & Messzeit  t/s & Zählrate  M & Netto Zählrate n in 1/s \\ \hline
0,5       & 80            & 8308        & 103,85                  \\ \hline
1,1       & 80            & 4786        & 59,825                  \\ \hline
1,5       & 80            & 2283        & 28,5375                 \\ \hline
2,0       & 80            & 915         & 11,4375                 \\ \hline
2,5       & 80            & 113         & 1,4125                  \\ \hline
\end{tabular}
\end{table}

\end{document}