% !TeX root = ..\protokoll.tex
\documentclass[../protokoll.tex]{subfiles}
\graphicspath{{\subfix{../images/}}}

\begin{document}
\part{Einleitung}
\begin{messageBox}{orange}{Hinweis}
Diese Versuche beschäftigen sich mit Radioaktivität. Da das Arbeiten mit
Radioaktivität schädlich für Personen und andere Lebewesen ist, müssen bei den
Versuchen die "`AAA des Strahlenschutzes:"'
\begin{itemize}
    \item Verminderung der \textsc{Aufenthaltsdauer} im Strahlungsfeld
    \item Vergrößerung des \textsc{Abstandes} der exponierten Person zur Strahlenquelle
    \item Verminderung der Personendosis durch \textsc{Abschirmungen} des Strahlenfeldes
\end{itemize}
beachtet werden. (vgl. \cite{AAA-Regel})

Des Weiteren sollten schwangere Personen diese Versuche nicht
durchführen, da für das ungeborene Kind eine besondere Gefahr besteht.
\end{messageBox}

In den folgenden Versuchen werden elementare Grundlagen für die Messung von
radioaktiver Strahlung und ihrer Wechselwirkung mit Materie behandelt.

Im Versuch "`\nameref{sec:Abstandsgesetz}"' geht es um die Verminderung der
Strahlenintensität mit zunehmenden Abstand von der Strahlungsquelle. Dieses
Gesetz hat im Strahlenschutz eine große Bedeutung.

In den weiteren Versuchen "`\nameref{sec:Abschwächung Gamma-Strahlung}"' und
"`\nameref{sec:Abschwächung Beta-Strahlung}"' widmet sich der quantitativen
Messung von materialabhängigen Abschwächungsvorgängen, da auch diese im Bereich
des Strahlenschutzes eine weitere große Bedeutung hat.

\begin{messageBox}{blue}{Hinweis}
    Nach \cite{listeEntfallendeVersuche} entfallen aus dem Skript die folgenden
    Versuche und Versuchsteile:
    \begin{itemize}
        \item Kapitel 3.5: Abstandsgesetz für \isotope[90]{Sr}
        \item Kapitel 3.8
    \end{itemize}
\end{messageBox}
\end{document}

