% !TeX root = ..\protokoll.tex
\documentclass[../protokoll.tex]{subfiles}
\graphicspath{{\subfix{../images/}}}
\begin{document}
\part{Einleitung}
Beim Entwurf und der Realisierung optischer Instrumente und Experimente hat die geometrische Optik
(Strahlenoptik) nach wie vor eine große praktische Bedeutung. Sie beruht auf vier Gesetzen, die sich aus
dem \textsc{Fermat}schen Prinzip ableiten lassen: 
\begin{itemize}
    \item der Geradlinigkeit der Lichtausbreitung
    \item der Umkehrbarkeit optischer Wege
    \item dem Reflekttionsgesetz
    \item dem Brechungsgesetz
\end{itemize}
Neben soliden Kenntnissen der recht
elementaren theoretischen Grundlagen der geometrischen Optik sind vor allem praktische Erfahrungen im
richtigen Umgang mit einfachen optischen Komponenten nützlich, die in diesem Versuch gewonnen
werden.

\begin{messageBox}{green}{Schreibweisen und Vereinbarungen}
    \begin{itemize}[noitemsep,leftmargin=*]
        \item Bei Abbildungen und Beschreibungen von Strahlungsgängen, breitet
                sich das Licht grundsätzlich von links nach rechts aus
        \item Der Raum links von einem abbildenden optischen System nennt sich
                Gegenstandsraum, der Raum rechts davon Bildraum
        \item Die Achse, die durch die Linsenmitte und die Brennpunkte einer
                Linse verläuft, heißt optische Achse. In Abbildungen wird
                diese durch eine horizontale, strichpunktierte Linie veranschaulicht.
        \item Aus den Näherungen des Brechungsgesetzes ergibt sich für einen Lichtstrahl,
                der aus einem Medium mit der Brechzahl $n_1$ in ein weiteres Medium mit
                der Brechzahl $n_2$ eintritt, folgender Zusammenhang:
                \begin{equation}\label{eq:Verhältnis Brechzahlen}
                    n_1 \sin \alpha = n_2 \sin \beta
                \end{equation}
    \end{itemize}

    
\end{messageBox}

\begin{messageBox}{blue}{Hinweis}
    Nach \cite{listeEntfallendeVersuche} entfallen aus dem Skript keine Versuchsteile.

    Die Fragen aus dem Theorieteil entfallen weiterhin.
\end{messageBox}
\end{document}