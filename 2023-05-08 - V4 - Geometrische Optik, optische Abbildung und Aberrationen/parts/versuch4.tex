% !TeX root = ..\protokoll.tex
\documentclass[../protokoll.tex]{subfiles}
\graphicspath{{\subfix{../images/}}}
\begin{document}
\section{Sehwinkelvergrößerung mit einem Fernrohr}
In diesem Versuch wird die Sehwinkelvergrößerung $M_s$ an einem Fernrohr bestimmt. Dazu wird zuerst der Sehwinkel $\alpha_0$ mit dem bloßen Auge bestimmt, indem von einem Abstand $l$=3 m von der Fensterscheibe auf die Aluminiumschiene mit zwei Markierungen geschaut wird. Dann wird der scheinbare Abstand $s_0$ der Markierungen auf dem Fenster markiert. Dann gilt für den Winkel:
\begin{equation}
    \alpha_0 \approx \frac{s_0}{l}
\end{equation}
Bei der Messung ergibt sich für $s_0$=3,5 cm und somit ein $\alpha_0=0,012°$.
Dann wird die Messung mit dem Teleskop wiederholt. Dieses besteht aus 2 Linsen mit Brennweiten $f_1=(1000\pm10) mm$ und $f_2=(100\pm1) mm$ .
Für diese Messung ergibt sich der Abstand $s_1$=30 cm und für den Winkel $\alpha_1=0,1°$.
Damit ergibt sich für $M_s=\frac{\alpha_1}{\alpha_0}$=8,57. Nach der Formel:
\begin{equation}
    M_s=\frac{f_1}{f_2}
\end{equation}
ergibt sich für $M_s=10\pm 0,2$. Dies liegt überhalb des zuvor berechneten Wertes, jedoch trotzdem noch in ähnlicher Größenordnung. Mögliche Gründe könnten sein, dass die erste Formle nur eine Abschätzung für $M_s$ ist und es somit zu Ungenauigkeiten kommen kann.
\end{document}